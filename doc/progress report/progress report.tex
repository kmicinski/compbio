\documentclass[10pt]{article}
\usepackage[margin=0.75in]{geometry}
\usepackage{amsmath, amsthm, amssymb}
\usepackage{qtree}
\usepackage{proof}
\usepackage{url}
\usepackage{ulem}
%\usepackage{biblatex}
%\bibliography{progress report}

\title{RNA-seq in Julia}
\author{James Parker}
\author{Matthew Mauirello}
\author{Kristopher Micinski}

\begin{document}
\maketitle

\section{Progress Report}

\subsection{Abstract}

Traditionally, genomics research has used loosely typed, dynamic languages 
(such as R) for analyzing and processing data.  These languages are less 
performant than traditional general purpose programming languages (such as Java or C++), 
but offer more flexibility for scriptable (interactive) processing.   Recently, the Julia language 
has emerged as a best of both worlds: a strong nominal type system with minimal syntax 
(allowing for concise scripting, appropriate for research settings) and efficient performance (as it is JIT 
compiled to LLVM).  However, although this language provides a rich type system to express 
biological research, the main selling point of dynamic languages (such as R) remains the prevalence 
of libraries to analyze data.

Our project adds to Julia a library to analyze genomic data and perform numerous computations 
germane to research. Our library allows:

\begin{enumerate}
\item The representation of genomic ranges and intervals, including calculation of 
union intersection genes, and other functions to manipulate these constructs.
\item Parsing of BAM files, allowing them to be held as memory mapped representations 
mentioned in point 1.
\item Calculation of various genomic functions, such as overlap regions between sets of 
genomic range data.
\end{enumerate}

Our library is implemented as an extension to the currently existing Julia facilities for 
dealing with biological data: the \texttt{BioSeq} library \footnote{https://github.com/diegozea/BioSeq.jl}.  
While \texttt{BioSeq} currently supports some biological primitives (such as representing DNA and proteins) 
and operations using them (converting DNA to RNA, reverse complement, etc), it 
does not include support for calculating union intersection genes.  We also implement 
parsing BAM files into Julia objects, by using Julia's FFI capabilities to do parsing using the 
SAMtools library\footnote{http://samtools.sourceforge.net/}.

Currently, each of the group members has read the Julia language manual, and we have contacted 
the Julia developers to inform them of the project's directions and solicit help getting a working installation 
of the latest (v0.2) Julia branch\footnote{The current language includes a number of useful facilities, such as 
immutable values, memory mapped arrays, etc...}.  We have began to read about FFI and look into using 
SAMtools to parse BAM files (using the files from the homework as an example) and have started writing 
type signatures for basic operations on genomic ranges values (represented by the \texttt{GRanges} object 
in BioConductor).

\section{Existing Work}

Currently, the Julia language has relatively few packages.  It does
include a \texttt{BioSeq} \cite{bioseq} package, but it
only includes limited functionality for biological computing.  
Other libraries exist for performing biological computations in other languages.
A few examples of these libraries include the \texttt{BioConductor} \cite{bioseq} package for R and the \texttt{BioPerl} \cite{bioperl}  package for Perl. Do to familiarity, we use \texttt{BioConductor} as our primary comparison to our implementation in Julia.

% KRIS: http://www.bioperl.org/wiki/Main_Page

\section{Methods and tools}

Our tools used during development include Julia version 0.2 and the IDE JuliaStudio. We will also leverage applicable features from the current \texttt{BioSeq} package. We may also end up using functionality provided by SAMtools to parse BAM files.
Upon completing our implementation, we will use the data from homework 1. With this data, we will compare our implementations' output with that of \texttt{BioConductor}'s to verify correctness. We will also perform basic benchmarking to rank the performance of each library.

%
%Our project explores the Julia language for scientific computing, and
%adds support for computing with RNA sequencing data.
%
%Currently, the Julia language has relatively few packages.  It does
%include a \texttt{BioSeq} \cite{bioseq} package, but it appears to
%only include limited functionality for biological computing.  In our
%project, we will first explore this library and determine its
%applicability to problems from homework examples: specifically the
%RNA-seq problems from homework 1.  While doing so, we will implement
%additional features not currently implemented in \texttt{BioSeq}, which
%are required to complete the homework problems.
%
%We anticipate adding support for types such as \texttt{GRanges} and
%implementing BAM file parsing to Julia.  The result of our project will be a
%solution manual to homework questions, as well as the Julia library
%(with accompanying documentation).  Our Julia code will be developed
%in a public\footnote{\url{https://github.com/kmicinski/compbio}} git
%repository.
%
%To study the Julia language, we will read the public documentation and
%associated papers \cite{juliapaper}.  Because we are replicating part
%of the \texttt{BioConductor} package in R \cite{bioconductor}, we will
%also read the associated documentation for the necessary classes.
%
%To test that our libraries are correct, we will develop both a set of
%unit tests and check our solutions for the homework against the
%correct solutions for homework 1.  We will also benchmark our 
%implementation to compare performance between Julia and R. 
%We will use the same data as used for the homework.
%
\section{Updated Timeline}

\begin{description}
\item[April $5^{th}$] \sout{Read all of the references (Julia paper) and work
  through relevant examples from Julia tutorials given on the homepage
  and elsewhere.}

\item[April $15^{th}$] \sout{Changed scope of project to implement memory mapped BAM file parsing, represent genomic ranges, and perform overlap querying. }

\item[April $19^{th}$] \sout{Progress report completed.}

\item[April $26^{th}$] Preliminary implementation of the project
  complete. May not have all unit tests and documentation written.  

\item[May $1^{th}$] Documentation and unit tests for the project
  completed.  Polished solutions to the homework packaged, as well as
  code documented.

\item[May $6^{th}$] Project presentation complete.

\item[May $10^{th}$] Final project writeup packaged and distributed,
  including webpage to describe project.

\end{description}

\end{document}


