\documentclass[10pt]{article}
\usepackage[margin=0.75in]{geometry}
\usepackage{amsmath, amsthm, amssymb}
\usepackage{qtree}
\usepackage{proof}
\usepackage{url}
\usepackage{biblatex}
\bibliography{proposal}

\title{RNA-seq in Julia}
\author{James Parker}
\author{Matthew Mauirello}
\author{Kristopher Micinski}

\begin{document}
\maketitle

\section{Project Description}

Our project explores the Julia language for scientific computing, and
adds support for computing with RNA sequencing data.

Currently, the Julia language has relatively few packages.  It does
include a \texttt{BioSeq} \cite{bioseq} package, but it appears to
only include limited functionality for biological computing.  In our
project, we will first explore this library and determine its
applicability to problems from homework examples: specifically the
RNA-seq problems from homework 1.  While doing so, we will implement
additional features not currently implemented in \texttt{BioSeq}, which
are required to complete the homework problems.

We anticipate adding support for types such as \texttt{GRanges} and
implementing BAM file parsing to Julia.  The result of our project will be a
solution manual to homework questions, as well as the Julia library
(with accompanying documentation).  Our Julia code will be developed
in a public\footnote{\url{https://github.com/kmicinski/compbio}} git
repository.

To study the Julia language, we will read the public documentation and
associated papers \cite{juliapaper}.  Because we are replicating part
of the \texttt{BioConductor} package in R \cite{bioconductor}, we will
also read the associated documentation for the necessary classes.

To test that our libraries are correct, we will develop both a set of
unit tests and check our solutions for the homework against the
correct solutions for homework 1.  We will also benchmark our 
implementation to compare performance between Julia and R. 
We will use the same data as used for the homework.

\section{Timeline}

\begin{description}
\item[April $5^{th}$] Read all of the references (Julia paper) and work
  through relevant examples from Julia tutorials given on the homepage
  and elsewhere.

\item[April $19^{th}$] Preliminary implementation of the project
  complete.  Able to complete the homework as given in the assignment.
  May not have all unit tests and documentation written.  Progress
  report completed.

\item[April $26^{th}$] Documentation and unit tests for the project
  completed.  Polished solutions to the homework packaged, as well as
  code documented.

\item[May $6^{th}$] Project presentation complete.

\item[May $10^{th}$] Final project writeup packaged and distributed,
  including webpage to describe project.

\end{description}

\end{document}


